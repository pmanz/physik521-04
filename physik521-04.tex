\input{header.tex}

\setcounter{section}{0}
\renewcommand\thesection{H\,4.\arabic{section}}
\renewcommand\thesubsection{\thesection.\alph{subsection}}

\title{physik521: Übungsblatt 04}
\author{%
    Lino Lemmer \\ \small{\texttt{s6lilemm@uni-bonn.de}}
    \and
    Martin Ueding \\ \small{\texttt{mu@martin-ueding.de}}
    \and
    Paul Manz \\ \small{\texttt{p.m@uni-bonn.de}}
}

\begin{document}
\maketitle

\section{Thermodynamische Relationen}

\section{Zentraler Grenzwertsatz}

\subsection{}

Mit der Wahrscheinlichkeit für das Ereignis $(x_1, \ldots, x_N)$ ist die
Wahrscheinlichkeit gemeint, mit der bei $N$ Ereignissen gerade diese in gerade
dieser Reihenfolge herauskommen? Dies ist dann:
\[
    \prod_{i=1}^N p(x_i)
\]

Bei der Wahrscheinlichkeit für ein bestimmtes $y$ kommt es auf die Reihenfolge
der $x_i$ nicht an. Daher können wir mit $N!$ multiplizieren, um alle möglichen
Permutationen zusammenzufassen.

Je nach der Struktur der Menge $M$ könnte es auch sein, dass verschiedene
Ergebnismengen $\set{x_i \colon i = [1, N] \cap \mathbb N}$ zum gleichen $y$
führen, aber das kann man nicht genau sagen.

Also erhalten wir:
\[
    P_N(y) = \frac{N!}{N} \prod_{i=1}^N p(x_i)
\]

\subsection{Charakteristische Funktion}

Wir nehmen die Definition der charakteristischen Funktion und setzen unser
$P_N$ dort ein:
\begin{align*}
    X_N(k)
    &= \int \exp{\mathrm i k \frac 1N \sum_{i=1}^N (x_i - \langle x \rangle)}
    \frac{N!}N \prod_{i=1}^N p(x_i) \dif x_i \\
    \intertext{%
        Die Summe im Exponenten kann man in ein Produkt von
        Exponentialfunktionen schreiben.
    }
    &= \frac{N!}N \prod_{i=1}^N \underbrace{\int 
        \exp{\mathrm i k \frac 1N (x_i - \langle x \rangle)}
    p(x_i) \dif x_i}_{\chi\del{\frac kN}} \\
    \intertext{%
        Produkte gleicher Faktoren sind Potenzen.
    }
    &= \frac{N!}N \del{\chi\del{\frac kN}}^N
\end{align*}

Dies stimmt bis auf unseren Vorfaktor auch mit dem Ergebnis überein, das wir
erhalten sollten.

\end{document}

% vim: spell spelllang=de tw=79
