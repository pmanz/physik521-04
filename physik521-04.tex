\input{header.tex}

\setcounter{section}{0}
\renewcommand\thesection{H\,4.\arabic{section}}
\renewcommand\thesubsection{\thesection.\alph{subsection}}

\title{physik521: Übungsblatt 04}
\author{%
    Lino Lemmer \\ \small{\texttt{s6lilemm@uni-bonn.de}}
    \and
    Martin Ueding \\ \small{\texttt{mu@martin-ueding.de}}
    \and
    Paul Manz \\ \small{\texttt{p.m@uni-bonn.de}}
}

\begin{document}
\maketitle

\section{Thermodynamische Relationen}
Es gilt zu zeigen:
\[\frac{c_p}{c_V}=\frac{\kappa_T}{\kappa_S} \]
Setzen zunächst die Definitionen ein:
\begin{align*}
\frac{c_p}{c_V}=\frac{T \del{\dpd ST}_{p}}{T \del{\dpd ST}_V}= \frac{ \del{\dpd ST}_{p}}{ \del{\dpd ST}_V}\\
\end{align*}

In der Ableitung im Zähler wird $p=p\del{S,T}$ festgehalten, es gilt deshalb:
\begin{align*}
\dif p = \del{\dpd pT}_S \dif T + \del{\dpd pS}_T \dif S = 0 \\
\implies \del{\dpd ST}_p = -\frac{\del{\dpd pT}_S}{\del{\dpd pS}_T}
\end{align*}

Im Nenner wird $V=V\del{S,T}$ festgehalten, also:
\begin{align*}
\dif V = \del{\dpd VS}_T \dif S + \del{\dpd VT}_S \dif T = 0 \\
\implies \del{\dpd ST}_V = -\frac{\del{\dpd VT}_S}{\del{\dpd VS}_T} \\
\end{align*}

Wir erhalten als Zwischenergebnis:
\begin{align*}
\frac{c_p}{c_V}=\frac{\del{\dpd pT}_S}{\del{\dpd VT}_S} \frac{\del{\dpd VS}_T}{\del{\dpd pS}_T}
\end{align*}

Nach der Kettenregel gilt:
\begin{align*}
\del{\dpd pV}_S \del{\dpd VT}_S &= \del{\dpd pT}_S \\
\implies \frac{\del{\dpd pT}_S}{\del{\dpd VT}_S} &= \del{\dpd pV}_S =\frac{1}{\kappa_S} \\
\del{\dpd Vp}_T \del{\dpd pS}_T &= \del{\dpd VS}_T \\
\implies \frac{\del{\dpd VS}_T}{\del{\dpd pS}_T} &= \del{\dpd Vp}_T =\kappa_T \\
\implies \frac{c_p}{c_V} &= \frac{\kappa_T}{\kappa_S}
\end{align*}

\section{Zentraler Grenzwertsatz}

\subsection{}

Mit der Wahrscheinlichkeit für das Ereignis $(x_1, \ldots, x_N)$ ist die
Wahrscheinlichkeit gemeint, mit der bei $N$ Ereignissen gerade diese in gerade
dieser Reihenfolge herauskommen? Dies ist dann:
\[
    \prod_{i=1}^N p(x_i)
\]

Bei der Wahrscheinlichkeit für ein bestimmtes $y$ kommt es auf die Reihenfolge
der $x_i$ nicht an. Daher können wir mit $N!$ multiplizieren, um alle möglichen
Permutationen zusammenzufassen.

Je nach der Struktur der Menge $M$ könnte es auch sein, dass verschiedene
Ergebnismengen $\set{x_i \colon i = [1, N] \cap \mathbb N}$ zum gleichen $y$
führen, aber das kann man nicht genau sagen.

Also erhalten wir:
\[
    P_N(y) = \frac{N!}{N} \prod_{i=1}^N p(x_i)
\]

\subsection{Charakteristische Funktion}

Wir nehmen die Definition der charakteristischen Funktion und setzen unser
$P_N$ dort ein:
\begin{align*}
    X_N(k)
    &= \int \exp{\ii k \frac 1N \sum_{i=1}^N \del{x_i - \mw{x}}}
    \frac{N!}N \prod_{i=1}^N p(x_i) \dif x_i \\
    \intertext{%
        Die Summe im Exponenten kann man in ein Produkt von
        Exponentialfunktionen schreiben.
    }
    &= \frac{N!}N \prod_{i=1}^N \underbrace{\int
        \exp{\ii k \frac 1N \del{x_i - \mw{x}}}
    p(x_i) \dif x_i}_{\chi\del{\frac kN}} \\
    \intertext{%
        Produkte gleicher Faktoren sind Potenzen.
    }
    &= \frac{N!}N \del{\chi\del{\frac kN}}^N
\end{align*}

Dies stimmt bis auf unseren Vorfaktor auch mit dem Ergebnis überein, das wir
erhalten sollten.

\subsection{Grenzfall $N\to\infty$}

Wir rechnen mit dem gegebenen Ergebnis weiter:
\begin{align*}
    X_N(k) &= \del{\chi\del{\frac{k}{N}}}^N \\
           &= \exp{N\log\del{\chi\del{\frac{k}{N}}}}
    \intertext{Taylrn von $\log\del{\chi}$ um $\frac{k}{N} = 0$:}
    \log\del{\chi\del{\frac{k}{N}}} &= \log\del{\int\dif y\, \exp{\ii
    \frac{k}{N}\del{y-\mw{y}}}P_N(y)} \\
    &= \log\del{\int\dif y\, P_N(y)} + \frac{P_N(y)}{\del{\int\dif y
    P_N(y)}}\frac{k}{N} + \dots
\end{align*}
\end{document}

% vim: spell spelllang=de tw=79
